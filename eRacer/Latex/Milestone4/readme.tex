\documentclass[11pt]{article}
\usepackage[utf8]{inputenc}
\usepackage[T1]{fontenc}
\usepackage[pdftex]{hyperref} 
\usepackage{graphicx}
\usepackage{amsmath}
\usepackage{float}
\usepackage{palatino}
\usepackage{multirow}
\usepackage{longtable}
\usepackage{underscore}
\restylefloat{figure}

\setlength\oddsidemargin{4mm}
\setlength\evensidemargin{4mm}
\setlength\textwidth{16cm}

\usepackage{fancyhdr}

\usepackage{setspace}


\onehalfspacing

\pagestyle{fancy}
% with this we ensure that the chapter and section
% headings are in lowercase
\renewcommand{\sectionmark}[1]{\markboth{\thesection.\ #1}{}}



\hypersetup 
{% 
pdftitle = {eRacerX Milestone 4 Readme}, 
pdfauthor = {Michael Blackadar, Tom Flanagan, Don Ha, Ole Rehmsen, John Stuart}, 
pdfkeywords = {eRacerX, racing, boost},
colorlinks={false},
linkbordercolor={1 1 1}, % set to white
citebordercolor={1 1 1}, % set to white 
urlcolor=white % set to white 
} 

\setcounter{tocdepth}{2}

\begin{document}

\begin{titlepage}
\begin{center}
% textsc is small caps
\textsc{
%
University of Calgary\\
Faculty of Science\\
Department of Computer Science\\ 
}
\vspace{3cm}
%
% the title
%
{
\includegraphics[width=12cm]{img/eRacerX-Logo.png}\\
\textsc{
\Huge
Milestone 4 Readme
}
}
\vspace{5mm}


% \today is the compilation date
\vspace{5mm}
{
 \today \\
 \ \\
 }
%
\vfill % fill vertical space
%
% Bottom of the page
%
\begin{flushright} 
Michael Blackadar \\
\vspace{3mm}
Tom Flanagan\\
\vspace{3mm}
Don Ha\\
\vspace{3mm}
Ole Rehmsen\\
\vspace{3mm}
John Stuart
\end{flushright}



\end{center}
\end{titlepage}

\label{introduction}
\section{Introduction}

This document is meant to provide some information related to the release of milestone 4 of the space racing simulation \emph{eRacerX} by \emph{Wallsocket Studios}. Section \ref{new-features} discusses a selection of new features in milestone 4. Section \ref{known-issues} lists some bugs that we have encountered and were not able to fix on time. Section \ref{future-work} outlines our plans for final product.

General information that has been revised since the readme for milestone 2 such as the controls (Section \ref{controls}), building from source and running  the game (Section \ref{building-from-source}) has been moved to the appendix at the end of the document. 


\section{New features}
\label{new-features}

Here is a list of some of the new features since milestone 3 that in our opinion deserve explicit mentioning.

\subsection{Boost}

Boost will now not only let you go faster, it will also give you a small initial hop that can be used to clear meteors. The star effect has been improved for boosting to make for a better sense of speed. Boost also recharges significantly slower since the players additionally have the opportunity to steal boost.

\subsubsection{Boost stealing}

Players can now steal each others boost by driving up behind another vehicle. The stealing is automatic if one is close enough. A simple laser beam indicates whether or not you are currently stealing. One can steal until the other player has no boost left to steal.



\subsection{Multiplayer}

\emph{eRacerX} now features a split-screen multiplayer. There can be 1,2 or 4 human players which can chose from one of 2 keyboard mappings and 4 controllers in the race setup menu. 

\subsection{Advanced menu}

There is now a more advanced menu that lets you chose the color and controls of the players as well as the number of AIs to use. There is also the option to go back to main menu from paused state or even to restart the race with the same settings as the current race.

\subsection{Hud}

We have added a number of hud elements to display information about the game: In addition to the lap counter, the players position is displayed. There is a simple boost bar showing how much boost fuel is available and interface element showing how far each vehicle has proceeded in the lap. There is an indicator if the vehicle is facing the wrong way. There is also a personal finish screen to inform players in multiplayer that they have finished.

\subsection{More content}

To offer a more immersive experience, we have added additional sounds for meteor collisions as well as in the menu. Textures have been improved to have a grungier appearance and more car colors are available. Sections of the tracks are now surrounded by glass rings to provide more variety and a reference point that improves the sense of speed.

\subsection{Driving code in C++}

Porting the vehicle to C++ led to a more stable integration of the physics and thereby allowed us to further speed up vehicles. 

\subsection{Improved AI}

The AI now dodges meteors and vehicles instead just driving straight and following the race line. This is especially noticable at the start where they used to drive through vehicles in their way. They also use boost whenever they are on a straight section of the track and are not running danger of jumping off the edge. 

\subsection{Improved track}

We have modified the second track by compressing the exciting features on a shorter overall lap length to make driving less monotone.


\section{Known issues}
\label{known-issues}


\subsection{Support for variable screen resolution and aspect ratios}

Currently, most of the hud and menu rendering depends on the resolution of the rendering surface. This is noticable if one plays two player multiplayer, where the aspect ratio is 8:3 instead of 4:3. Also the current resolution of 800 by 600 is relatively low and should be selectable from the menu instead. Support for full screen rendering would also be desirable.
This has not been fixed yet due to a driver bug in DirectX, that results in sprites being rendered in the viewport on most graphics cards but not in others (such as the nVidia 8600M GT). 

\subsection{Sound bug}

On some machines we had issues with sound popping, but not on others. This does not seem to correlate with the overall power of the machines or the frame rate. We will have to explore this in future. 

\subsection{Alpha blending order}

The alpha blending for the blob shadow is not working properly if one flies off the track because the rendering is not happening in the correct order. This will be fixed in final version of the game (promise:-).

\section{Future work}
\label{future-work}

In this section we want to touch on a few topics that we want to work on for the final product. The last weeks will be used predominantly for bug fixing, beautification and game play tweaks.

\subsection{Beautification}

We want to have more props in the game to make the world look more vivid. The graphics of the hud will be revised. We want to add more visual effects, such as the partially transparent track (depends on alpha sorting), an animated boost stealing beam and particle effects for both the rocket propulsion and when the vehicle collides with the walls. 

\subsection{Gameplay tweaks}

\subsubsection{Even smarter AI}

The AI should be less predictable and more dynamic. It should also use some rubber banding to ensure it does not get too far ahead, or behind. The AI also currently does not actually try to steal the players boost. This can be improved.

\subsubsection{Discontinuous track and jumps}

At this point a track is a closed b-spline. We would like to experiment with multiple open b-spline sections that would allow us to have jumps. This will only be used if it turns out to make the game more fun.


\appendix

\section{Controls}
\label{controls}

In this section we list the controls for our game. There are different controls
depending on whether you are in the actual game or in a menu. 

\subsection{In-game controls}

\begin{center}
\begin{longtable}{lcc}
\caption{In-game controls} \label{ingame-controls-table} \\

%This is the header for the first page of the table...
\hline \hline \\[-2ex]
   \multicolumn{1}{c}{\textbf{Command}} &
   \multicolumn{1}{c}{\textbf{Keyboard1}} &
   \multicolumn{1}{c}{\textbf{Keyboard2}} &
   \multicolumn{1}{c}{\textbf{Gamepad}} \\[0.5ex] \hline
   \\[-1.8ex]
\endfirsthead

%This is the header for the remaining page(s) of the table...
\multicolumn{3}{c}{{\tablename} \thetable{} -- Continued} \\[0.5ex]
  \hline \hline \\[-2ex]
  \multicolumn{1}{c}{\textbf{Command}} &
  \multicolumn{1}{c}{\textbf{Keyboard1}} &
  \multicolumn{1}{c}{\textbf{Keyboard2}} &
  \multicolumn{1}{c}{\textbf{Gamepad}} \\[0.5ex] \hline
  \\[-1.8ex]
\endhead

%This is the footer for all pages except the last page of the table...
  \multicolumn{3}{l}{{Continued on Next Page\ldots}} \\
\endfoot

%This is the footer for the last page of the table...
  \\[-1.8ex] \hline \hline
\endlastfoot

%Now the data...
Accelerate & W & Up & Right Trigger \\ 
Reverse & S & Down & Left Trigger \\
Break & Left Shift & Right Control & B \\ 
Steer left & A & Left & \multirow{2}{*}{Left Analog Stick} \\ 
Steer right & D & Right \\
Boost & Space & Numpad 0 & A \\
Cycle cameras & C & Numpad Decimal & Y \\
Pause menu & Esc & Pause key & Start \\
\end{longtable}
\end{center}

\subsection{Menu controls}

\begin{center}
\begin{longtable}{lcc}
\caption{In-game controls} \label{ingame-controls-table} \\

%This is the header for the first page of the table...
\hline \hline \\[-2ex]
   \multicolumn{1}{c}{\textbf{Command}} &
   \multicolumn{1}{c}{\textbf{Keyboard}} &
   \multicolumn{1}{c}{\textbf{Gamepad}} \\[0.5ex] \hline
   \\[-1.8ex]
\endfirsthead

%This is the header for the remaining page(s) of the table...
\multicolumn{3}{c}{{\tablename} \thetable{} -- Continued} \\[0.5ex]
  \hline \hline \\[-2ex]
  \multicolumn{1}{c}{\textbf{Command}} &
  \multicolumn{1}{c}{\textbf{Keyboard}} &
  \multicolumn{1}{c}{\textbf{Gamepad}} \\[0.5ex] \hline
  \\[-1.8ex]
\endhead

%This is the footer for all pages except the last page of the table...
  \multicolumn{3}{l}{{Continued on Next Page\ldots}} \\
\endfoot

%This is the footer for the last page of the table...
  \\[-1.8ex] \hline \hline
\endlastfoot

%Now the data...
Menu up & Up & \multirow{2}{*}{Left Analog Stick}\\ 
Menu down & Down \\ 
Select menu item & Enter & A \\ 
Go back & Esc & B \\
\end{longtable}
\end{center}


\section{Building from source}
\label{building-from-source}

\subsection{Prerequisites}

The following must be installed:
  
\begin{itemize}
  \item Microsoft Visual Studio 2008
  \item Microsoft DirectX SDK (August 2009)
  \item NVIDIA PhysX SDK v2.8.1
  \item Python 2.6
  \item SWIG-1.3.40
\end{itemize}  
  
\subsection{Environment variables}

The following environment variables
must be set (your paths may vary):

\begin{itemize}
  \item DXSDK_DIR = F:\textbackslash Program Files (x86)\textbackslash
  Microsoft DirectX SDK (August 2009)
  \item PHYSX_DIR = F:\textbackslash Program Files (x86)\textbackslash NVIDIA
  Corporation\textbackslash NVIDIA PhysX SDK\textbackslash v2.8.1
  \item PYTHON_DIR= F:\textbackslash Python26\textbackslash
  \item SWIG_DIR  = P:\textbackslash swigwin\item 
\end{itemize}  
 
\subsection{Building}

Build the project eRacer/eRacer.sln.

\subsection{Running}  
  
To run the program, run \texttt{eRacer/run-release.py} or
\texttt{eRacer/run-debug.py} with python.


\end{document}
