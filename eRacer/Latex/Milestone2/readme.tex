\documentclass[11pt]{article}
\usepackage[utf8]{inputenc}
\usepackage[T1]{fontenc}
\usepackage[pdftex]{hyperref} 
\usepackage{graphicx}
\usepackage{amsmath}
\usepackage{float}
\usepackage{palatino}
\usepackage{multirow}
\usepackage{longtable}
\usepackage{underscore}
\restylefloat{figure}

\setlength\oddsidemargin{4mm}
\setlength\evensidemargin{4mm}
\setlength\textwidth{16cm}

\usepackage{fancyhdr}

\usepackage{setspace}


\onehalfspacing

\pagestyle{fancy}
% with this we ensure that the chapter and section
% headings are in lowercase
\renewcommand{\sectionmark}[1]{\markboth{\thesection.\ #1}{}}



\hypersetup 
{% 
pdftitle = {eRacerX Milestone 2 Readme}, 
pdfauthor = {Michael Blackadar, Tom Flanagan, Don Ha, Ole Rehmsen, John Stuart}, 
pdfkeywords = {eRacerX, racing, boost},
colorlinks={false},
linkbordercolor={1 1 1}, % set to white
citebordercolor={1 1 1}, % set to white 
urlcolor=white % set to white 
} 

\setcounter{tocdepth}{2}

\begin{document}

\begin{titlepage}
\begin{center}
% textsc is small caps
\textsc{
%
University of Calgary\\
Faculty of Science\\
Department of Computer Science\\ 
}
\vspace{3cm}
%
% the title
%
{
\includegraphics[width=12cm]{img/eRacerX-Logo.png}\\
\textsc{
\Huge
Milestone 2 Readme
}
}
\vspace{5mm}


% \today is the compilation date
\vspace{5mm}
{
 \today \\
 \ \\
 }
%
\vfill % fill vertical space
%
% Bottom of the page
%
\begin{flushright} 
Michael Blackadar \\
\vspace{3mm}
Tom Flanagan\\
\vspace{3mm}
Don Ha\\
\vspace{3mm}
Ole Rehmsen\\
\vspace{3mm}
John Stuart
\end{flushright}



\end{center}
\end{titlepage}

\label{introduction}
\section{Introduction}

This document is meant to provide some information related to the release of
milestone 2 of the space racing simulation \emph{eRacerX} by \emph{Wallsocket}
Studios.

Here is an outline of the contents: Section \ref{controls} lists the keyboard and gamepad controls available in \emph{eRacerX}. Section \ref{building-from-source} explains how to build \emph{eRacerX} from source using \emph{Microsoft Visual Studio 2008}. Section \ref{design-decisions} discusses some of the decisions we have made concerning the development of \emph{eRacerX} and changes to the initial schedule.

\label{controls}
\section{Controls}

In this section we list the controls for our game. There are different controls
depending on whether you are in the actual game or in a menu. 

\subsection{In-game controls}

\begin{center}
\begin{longtable}{lcc}
\caption{In-game controls} \label{ingame-controls-table} \\

%This is the header for the first page of the table...
\hline \hline \\[-2ex]
   \multicolumn{1}{c}{\textbf{Command}} &
   \multicolumn{1}{c}{\textbf{Keyboard}} &
   \multicolumn{1}{c}{\textbf{Gamepad}} \\[0.5ex] \hline
   \\[-1.8ex]
\endfirsthead

%This is the header for the remaining page(s) of the table...
\multicolumn{3}{c}{{\tablename} \thetable{} -- Continued} \\[0.5ex]
  \hline \hline \\[-2ex]
  \multicolumn{1}{c}{\textbf{Command}} &
  \multicolumn{1}{c}{\textbf{Keyboard}} &
  \multicolumn{1}{c}{\textbf{Gamepad}} \\[0.5ex] \hline
  \\[-1.8ex]
\endhead

%This is the footer for all pages except the last page of the table...
  \multicolumn{3}{l}{{Continued on Next Page\ldots}} \\
\endfoot

%This is the footer for the last page of the table...
  \\[-1.8ex] \hline \hline
\endlastfoot

%Now the data...
Accelerate & W & Right Trigger \\ 
Break & S & Left Trigger \\ 
Steer left & A & \multirow{2}{*}{Left Analog Stick} \\ 
Steer right & D \\
Cycle cameras & C &  \\
Move flying camera & Arrow keys &  \\
Reload tuning parameters & R &  \\
Toggle debug info & Tab &  \\
Play jaguar sound & Spacebar &  \\
\end{longtable}
\end{center}

\subsection{Menu controls}

\begin{center}
\begin{longtable}{lcc}
\caption{In-game controls} \label{ingame-controls-table} \\

%This is the header for the first page of the table...
\hline \hline \\[-2ex]
   \multicolumn{1}{c}{\textbf{Command}} &
   \multicolumn{1}{c}{\textbf{Keyboard}} &
   \multicolumn{1}{c}{\textbf{Gamepad}} \\[0.5ex] \hline
   \\[-1.8ex]
\endfirsthead

%This is the header for the remaining page(s) of the table...
\multicolumn{3}{c}{{\tablename} \thetable{} -- Continued} \\[0.5ex]
  \hline \hline \\[-2ex]
  \multicolumn{1}{c}{\textbf{Command}} &
  \multicolumn{1}{c}{\textbf{Keyboard}} &
  \multicolumn{1}{c}{\textbf{Gamepad}} \\[0.5ex] \hline
  \\[-1.8ex]
\endhead

%This is the footer for all pages except the last page of the table...
  \multicolumn{3}{l}{{Continued on Next Page\ldots}} \\
\endfoot

%This is the footer for the last page of the table...
  \\[-1.8ex] \hline \hline
\endlastfoot

%Now the data...
Menu up & Up & \multirow{2}{*}{Left Analog Stick}\\ 
Menu down & Down \\ 
Select menu item & Enter & A \\ 
Go back & Esc & B \\
\end{longtable}
\end{center}


\label{building-from-source}
\section{Building from source}

\subsection{Prerequisites}

The following must be installed:
  
\begin{itemize}
  \item Microsoft Visual Studio 2008
  \item Microsoft DirectX SDK (August 2009)
  \item NVIDIA PhysX SDK v2.8.1
  \item Python 2.6
  \item SWIG-1.3.40
\end{itemize}  
  
\subsection{Environment variables}

The following environment variables
must be set (your paths may vary):

\begin{itemize}
  \item DXSDK_DIR = F:\textbackslash Program Files (x86)\textbackslash
  Microsoft DirectX SDK (August 2009)
  \item PHYSX_DIR = F:\textbackslash Program Files (x86)\textbackslash NVIDIA
  Corporation\textbackslash NVIDIA PhysX SDK\textbackslash v2.8.1
  \item PYTHON_DIR= F:\textbackslash Python26\textbackslash
  \item SWIG_DIR  = P:\textbackslash swigwin\item 
\end{itemize}  
 
\subsection{Building}

Build the project eRacer/eRacer.sln.

\subsection{Running}  
  
To run the program, run \texttt{eRacer/run-release.py} or
\texttt{eRacer/run-debug.py} with python.
\label{design-decisions}
\section{Design decisions}

In this section we will highlight some deviations from the design document
and the schedule distributed with milestone 1 of \emph{eRacerX}.

\subsection{Earlier gamepad}

We have decided to implement the gamepad support earlier than initially planned
so that we would have more time to tweak both keyboard and gamepad. At the
beginning we thought of the gamepad support as bonus feature that could be
implemented in the very end, instead we decided to fully support it early on to
meet different player's preferences.

\subsection{Later meteorids}

In contrast to our early plans we decided not to implement meteorids for
milestone 2, as we have found plenty of features that have a higher priority
and influence the driving more severely. We now plan to add meteorids as part
of milestone 3, which will be mostly about gameplay features.

\subsection{Triangle mesh track instead of B-Splines}

We decided not to generate the track automatically from b-spline control points
to give us more control over the features of the track and be able to use more
of the tools modern modeling software offers us. Using b-spline cruves would
make the modeling more restricted, as we would have had to model the b-spline,
than have it being displayed in some preview tool with the attributes annotated such as
props along the track side or wideness of the track. 

We are now using a simple trinangular mesh which allows us model the track in
as much detail as necessary and using all modeling tools, to make the track
more interesting. Loading and rendering also gets simpler. The downside of this
approach is that a triangular mesh is necessarily piecewise linear and more
consideration has to be spend on making the physics work smoothly.



\end{document}
