\documentclass[11pt]{article}
\usepackage[utf8]{inputenc}
\usepackage[T1]{fontenc}
\usepackage[pdftex]{hyperref} 
\usepackage{graphicx}
\usepackage{amsmath}
\usepackage{float}
\usepackage{palatino}
\usepackage{multirow}
\usepackage{longtable}
\usepackage{underscore}
\restylefloat{figure}

\setlength\oddsidemargin{4mm}
\setlength\evensidemargin{4mm}
\setlength\textwidth{16cm}

\usepackage{fancyhdr}

\usepackage{setspace}


\onehalfspacing

\pagestyle{fancy}
% with this we ensure that the chapter and section
% headings are in lowercase
\renewcommand{\sectionmark}[1]{\markboth{\thesection.\ #1}{}}



\hypersetup 
{% 
pdftitle = {eRacerX Readme}, 
pdfauthor = {Michael Blackadar, Tom Flanagan, Don Ha, Ole Rehmsen, John Stuart}, 
pdfkeywords = {eRacerX, racing, boost},
colorlinks={false},
linkbordercolor={1 1 1}, % set to white
citebordercolor={1 1 1}, % set to white 
urlcolor=white % set to white 
} 

\setcounter{tocdepth}{2}

\begin{document}

\begin{titlepage}
\begin{center}
% textsc is small caps
\textsc{
%
University of Calgary\\
Faculty of Science\\
Department of Computer Science\\ 
}
\vspace{3cm}
%
% the title
%
{
\includegraphics[width=12cm]{img/eRacerX-Logo.png}\\
\textsc{
\Huge 
Readme
}
}
\vspace{5mm}


% \today is the compilation date
\vspace{5mm}
{
 \today \\
 \ \\
 }
%
\vfill % fill vertical space
%
% Bottom of the page
%
\begin{flushright} 
Michael Blackadar \\
\vspace{3mm}
Tom Flanagan\\
\vspace{3mm}
Don Ha\\
\vspace{3mm}
Ole Rehmsen\\
\vspace{3mm}
John Stuart
\end{flushright}



\end{center}
\end{titlepage}

\label{introduction}
\section{Introduction}

This document is meant to provide some information related to the release of
the space racing simulation \emph{eRacerX} by \emph{Wallsocket Studios}. The
background story can be found in Section \ref{story}. Section
\ref{game-concepts} explains some game concepts that are unique to
\emph{eRacerX}. Section \ref{new-features} discusses a selection of new
features in the release version. General information such as the controls (Section \ref{controls}), building from source and running the game (Section \ref{building-from-source}) can be
found in the appendix at the end of the document.

\section{Story}
\label{story}
It is year 3728 and the galaxy has been swept up in racing fever! Recently the
finest and the fastest racers have set up the \emph{League of Extraordinary
Racers} to see who truly is the best racer in the galaxy. There can be only
one! You have been training your entire life for the opportunity to join the
league. Due to an ``unfortunate'' accident the league has many free spots for
up and coming racers such as yourself. You travel to the \emph{Xyzliat System}
where the league has set up the ultimate race. Big money and big prizes  are up
for stake, as well as a spot in the league. Your destiny awaits!

\section{Game concepts}
\label{game-concepts}

\emph{eRacerX} is a racing game set in space. The player can compete against
other human or ai racers on one of multiple tracks. 

\subsection{Boost}

Each car has a boost ability that dramatically increases its acceleration,
however, the boost fuel is limited, so that it can be engaged only for
short amounts of time.

There are two ways to recharge boost fuel:

\begin{enumerate}
  \item Boost fuel slowly recharges over time
  \item Boost fuel is also stolen automatically from other racers directly in
  front. This method is much faster. If a car steals from another, this can be
  seen by the energy beams streaming from the victim to the stealer.
\end{enumerate}

\subsection{Warping}

All cars include a warping capability that allows them to reset to the track
whenever the float off into space. This can be either done manually or happens
automatically if the car registeres that it lost traction to the track.


\section{New features}
\label{new-features}

Here is a list of features added since the last Milestone that in our opinion
deserve explicit mentioning.

\subsection{Transparent track}

The track is now made of a new transparent material that allows you to see
through and enjoy the nebula scenery.

\subsection{Psychedelic rainbow rings}

Rings around the track, pulsating in the colors of the rainbow make the game
more colorful and provide additional sense of speed by moving towards you.

\subsection{Gameplay tweaks}

Additional traction while boosting makes it easier to stay on the track at high
speeds and allows you to ram non boosting opponents.

Boost can now be stolen over a larger distance.

\subsection{AI tweaks}

AIs now have a rubber banding behavior that lets them slow down as they are
pulling too far ahead of human players.

The AI dodging mechanism has been revised using a randomized algorithm to
reduce the likeliness that they get stuck. It also makes sure that AIs act
diverse in situations where many cars drive close together.

\subsection{Highscores}

The game now keeps record of all finished races and displays local high scores
in the main menu. One can look at the high scores for each track and number of
laps, as well as the top lap time per track.

\subsection{More music}

We now have three different songs for different tracks. Also, the pause menu
has its own, calming music.

\subsection{Improved front end}

The game uses now a brushed metal front end for most menus as well as preview
images of the tracks. It also supports greyed out options and memorizes options
between multiple runs.

\subsection{More tracks}

\emph{eRacerX} now features a variety of tracks, some of which are mostly
straight and allow for top speeds, whereas others with numerous tight curves put
a higher demand on the players skill.

\appendix

\section{Controls}
\label{controls}

In this section we list the controls for our game. There are different controls
depending on whether you are in the actual game or in a menu. 

\subsection{In-game controls}

\begin{center}
\begin{longtable}{lccc}
\caption{In-game controls} \label{ingame-controls-table} \\

%This is the header for the first page of the table...
\hline \hline \\[-2ex]
   \multicolumn{1}{c}{\textbf{Command}} &
   \multicolumn{1}{c}{\textbf{Keyboard1}} &
   \multicolumn{1}{c}{\textbf{Keyboard2}} &
   \multicolumn{1}{c}{\textbf{Gamepad}} \\[0.5ex] \hline
   \\[-1.8ex]
\endfirsthead

%This is the header for the remaining page(s) of the table...
\multicolumn{3}{c}{{\tablename} \thetable{} -- Continued} \\[0.5ex]
  \hline \hline \\[-2ex]
  \multicolumn{1}{c}{\textbf{Command}} &
  \multicolumn{1}{c}{\textbf{Keyboard1}} &
  \multicolumn{1}{c}{\textbf{Keyboard2}} &
  \multicolumn{1}{c}{\textbf{Gamepad}} \\[0.5ex] \hline
  \\[-1.8ex]
\endhead

%This is the footer for all pages except the last page of the table...
  \multicolumn{3}{l}{{Continued on Next Page\ldots}} \\
\endfoot

%This is the footer for the last page of the table...
  \\[-1.8ex] \hline \hline
\endlastfoot

%Now the data...
Accelerate & W & Up & Right Trigger \\ 
Reverse & S & Down & Left Trigger \\
Break & Left Shift & Right Control & B \\ 
Steer left & A & Left & \multirow{2}{*}{Left Analog Stick} \\ 
Steer right & D & Right \\
Boost & Space & Numpad 0 & A \\
Reset & T & Right Shift & X \\
Cycle cameras & C & Numpad Decimal & Y \\
Pause menu & Esc & Pause key & Start \\
\end{longtable}
\end{center}

\subsection{Menu controls}

\begin{center}
\begin{longtable}{lcc}
\caption{Menu controls} \label{menu-controls-table} \\

%This is the header for the first page of the table...
\hline \hline \\[-2ex]
   \multicolumn{1}{c}{\textbf{Command}} &
   \multicolumn{1}{c}{\textbf{Keyboard}} &
   \multicolumn{1}{c}{\textbf{Gamepad}} \\[0.5ex] \hline
   \\[-1.8ex]
\endfirsthead

%This is the header for the remaining page(s) of the table...
\multicolumn{3}{c}{{\tablename} \thetable{} -- Continued} \\[0.5ex]
  \hline \hline \\[-2ex]
  \multicolumn{1}{c}{\textbf{Command}} &
  \multicolumn{1}{c}{\textbf{Keyboard}} &
  \multicolumn{1}{c}{\textbf{Gamepad}} \\[0.5ex] \hline
  \\[-1.8ex]
\endhead

%This is the footer for all pages except the last page of the table...
  \multicolumn{3}{l}{{Continued on Next Page\ldots}} \\
\endfoot

%This is the footer for the last page of the table...
  \\[-1.8ex] \hline \hline
\endlastfoot

%Now the data...
Menu up & Up & \multirow{2}{*}{Left Analog Stick}\\ 
Menu down & Down \\ 
Select menu item & Enter & A \\ 
Exit & Esc & B \\
\end{longtable}
\end{center}


\section{Building from source}
\label{building-from-source}

\subsection{Prerequisites}

The following must be installed:
  
\begin{itemize}
  \item Microsoft Visual Studio 2008
  \item Microsoft DirectX SDK (August 2009)
  \item NVIDIA PhysX SDK v2.8.1
  \item Python 2.6
  \item SWIG-1.3.40
\end{itemize}  
  
\subsection{Environment variables}

The following environment variables
must be set (your paths may vary):

\begin{itemize}
  \item DXSDK_DIR = F:\textbackslash Program Files (x86)\textbackslash
  Microsoft DirectX SDK (August 2009)
  \item PHYSX_DIR = F:\textbackslash Program Files (x86)\textbackslash NVIDIA
  Corporation\textbackslash NVIDIA PhysX SDK\textbackslash v2.8.1
  \item PYTHON_DIR= F:\textbackslash Python26\textbackslash
  \item SWIG_DIR  = P:\textbackslash swigwin\item 
\end{itemize}  
 
\subsection{Building}

Build the project eRacer/eRacer.sln.

\subsection{Running}  
  
To run the program, run \texttt{eRacer/run.py} with python. One can specify the
flags --release, --debug, --profile.


\end{document}
